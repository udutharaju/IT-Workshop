\documentclass{article}
\usepackage{amsmath}
\begin{document}
	The well known Pythagorean theorem \(x^2 +y^2 = z^2\) was proved to be invalid for other exponents
	Meaning the next equation has no integer solutions:The mass-energy equivalence is described by the famous equation\\
	\begin{math}
		E = mc^2\\
		E = m
	\end{math}\\
		
		Maths Equations Problems\\
		\[x^n + y ^n = z^n\]\\
		the formula of \\
		
		 \[(a+b)^2\]
		 	\begin{equation}
		 		a^2 + b^2 + 2ab = (a+b)^2 
		 \end{equation}

		 
		 \[(x+y)^2\]
		 \begin{equation}
		 	x^2 + Y^2 + 2xy = (x+y)^2
		 \end{equation}
	this is the simple math expression \(\sqrt{x^2+1}\)	inside text.
	And this is also the same:\\	\(\sqrt{x^2+1}\)\\
	but by using another command.\\
	This is a simple math expression without numbering	\[\sqrt{x^2+1}\]
	separates from text.\\
	This is also the same:\[\sqrt{x^2+1}\]\\\\\
	
	\title{QUESTION NUMBER THREE}
	QUESTION NUMBER THREE
	
\begin{align*}
	f(x) &= x^2\\
	g(x) &= \frac{1}{x}\\
	F(x) &= \int^a_b x^3
\end{align*}
\begin{gather}
	f(x) = ax^2 + bx +c\\
	g(x) = dx^3 + ex^2 + fx + g\\
	h(x) = \frac{1}{x}\\
	j(x) = \int^e_0 e^-t^2 dt
\end{gather}
\[
\begin{bmatrix}
	1 & 2 & 3\\
	1 & 2 & 3\\
	 1 & 2 & 3
\end{bmatrix}\]

	Some mathematical symbols: $\alpha$, $\beta$, $\sum_{i=1}^{n} x_i$,
	$\int_{a}^{b} f(x) \, dx$, $\lim_{x \to \infty} f(x)$.\\	
	
	@article{lamport1986latex,
		author = "Leslie Lamport",
		title = "LaTeX: A Document Preparation System",
		journal = "Addison-Wesley series in computer science",
		volume = "35",
		year
		= "1986",
		publisher = "Addison-Wesley"
	}z
		
\end{document}